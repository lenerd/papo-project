\documentclass[12pt,a4paper]{article}
\usepackage[utf8]{inputenc}
\usepackage{amsmath}
\usepackage{amsfonts}
\usepackage{amssymb}
\usepackage{graphicx}
\author{Lennart Braun, Armin Schaare, Theresa Eimer}
\title{Nugengo - Kann unser neuronales Netz besser Go spielen als wir}
\begin{document}

\maketitle
\tableofcontents

\section{Die Idee}
%Was es tun soll, wieso wir das machen, eine kurze Einleitung eben

\section{Über Go}
%Kurze Beschreibung der Regeln und unsere Umsetzung

\section{Unser Programm}

\subsection{Die einzelnen Bereiche}

\subsubsection{Das neuronale Netz}
%Der Aufbau des Netzes, Funktionsweise von Backpropagtion
\subsubsection{Go}
%Welche Komponenten? Wie spielt es zusammen?
\subsubsection{Der genetische Algorithmus}
%Wie funktioniert er?

\subsection{Die Tools}
%Welche gibt es? Wie setzt man sie ein? Warum macht das so Sinn?

\subsection{Parallelisierung und Optimierung}
%Wie wurde paralellisiert? Was wurde schneller? Wo wird am meisten Zeit verbraucht?

\section{Die Ergebnisse}
%Funktioniert es? Wie ist der Zeitaufwand? Was heißt das für Go mit neuroonalen Netzen?

\section{Fazit}
%Kurzer Schluss

\end{document}