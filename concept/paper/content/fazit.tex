\section{Fazit}

Die Spielleistung der trainierten neuronalen Netze war nicht überragend. Für
ein besseres Ergebnis in dieser Hinsicht, hätte man sich mehr mit der Theorie
hinter neuronalen Netzen beschäftigen müssen. Auch weitere Erfahrung in Bezug
auf Computer Go wäre sicherlich nützlich gewesen.

Da der Titel des Praktikum „Parallele Programmierung“ lautet, war die
Beschäftigung mit dem Thema der künstlichen Intelligenz nur ein Nebenprodukt.
Die im Vordergrund stehende Parallelisierung des Programms war insofern ein
Erfolg, dass wir ein naives, nur mäßig effizientes, Parallelisierungsschema
analysiert und die Schwachstellen erkannt haben.  Mittels den gewonnenen
Erkenntnissen wurde erfolgreich ein alternatives Schema entworfen und
implementiert. Dieses lieferte einen, in unseren Augen, akzeptablen
Speedup. Interessant an der Parallelisierung war, dass sich die zu verteilende
Last nicht vorhersehbar änderte und daher dynamisch, während der Laufzeit
darauf eingegangen werden musste.
