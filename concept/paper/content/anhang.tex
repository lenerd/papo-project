\begin{appendices}

\section{Verwendete Software}

Für unsere Anwendung haben wir die folgende Software verwendet:

Compiler
\begin{itemize}
    \item GNU Compiler Collection \cite{gcc}
\end{itemize}

Libraries
\begin{itemize}
    \item GNU C Library \cite{glibc}
    \item OpenMPI \cite{openmpi}
    \item GNU Readline \cite{readline}
    \item Check \cite{check}
\end{itemize}

Tools
\begin{itemize}
    \item Git \cite{git}
    \item CMake \cite{cmake}
    \item Doxygen \cite{doxygen}
    \item Vampir \cite{vampir}
\end{itemize}

Dieser Bericht wurde mit {\LaTeX} und {\KOMAScript} erstellt. Die Diagramme
wurden mit matplotlib \cite{matplotlib} erstellt.

\section{Dateiformat}

\lstinputlisting[%
    caption=Ausgabedatei von \texttt{ngg\_tool -a create -layout "2 3 2"{}
            -o filename},
    label=lst:fileformat,
]{content/code/fileformat}


\section{Command Line Interfaces}

\begin{lstlisting}[%
    basicstyle=\ttfamily,
    columns=flexible,
    keepspaces=true,
    caption=Interface von \texttt{ngg\_tool},
    label=lst:ngg_tool-help,
]
\$ ./ngg_tool --help
Usage: ngg_tool [OPTION...]

  -a, --action=STRING        create: creates a new neural network
                             gen-data: generates training data
                             train: trains a neural network with
                             supervised learning
                             calc: calculates output of a neural network
                             given input
  -b, --binary               use binary files
      --binary-in            use binary input file
      --binary-out           use binary output file
  -i, --in=FILE              load neuralnet(s) from file
  -l, --layer=NUMS           number of neurons in each layer
                             e.g. "2 3 3 2"
  -n, --number=NUM           number of networks to create or training
                             iterations
  -o, --out=FILE             output neuralnet(s) to file
  -s, --board-size=NUM       size of the used go board
  -t, --training-data=FILE   training data to use
  -v, --netver=0             nnet player version
  -?, --help                 Give this help list
      --usage                Give a short usage message
\end{lstlisting}

\begin{lstlisting}[%
    basicstyle=\ttfamily,
    columns=flexible,
    keepspaces=true,
    caption=Interface von \texttt{ngg\_game},
    label=lst:ngg_game-help,
]
\$ ./ngg_game --help
Usage: ngg_game [OPTION...]

  -b, --binary               use binary files
      --binary-in            use binary input file
      --binary-out           use binary output file
  -h, --human-readable       human readable output, no csv
  -i, --in=FILE              load neuralnet from file
  -n, --number=NUM           number of generations
  -o, --out=FILE             output neuralnet to file
  -s, --board-size=NUM       size of the used go board
      --sched-chunksize=int
      --sched-initial=double
      --seed=decimal         seed for the random number generator
  -v, --verbose              more prints more information
  -?, --help                 Give this help list
      --usage                Give a short usage message
\end{lstlisting}

\end{appendices}
