\section{Über Go}
%Kurze Beschreibung der Regeln und unsere Umsetzung
Go ist ein asiatisches Brettspiel, das normalerweise auf Brettern von $19
{\times} 19$ oder $9 {\times} 9$ Schnittpunkten gespielt wird. Es gibt zwei
Spieler, einer spielt schwarze, der andere weiße Steine. Nacheinander legt
zuerst der schwarze, dann der weiße Spieler, je einen Stein auf einen
Schnittpunkt. Ist ein Stein umzingelt, sind also alle seine vier direkt
angrenzenden Schnittpunkte von gegnerischen Steinen besetzt, so ist er
geschlagen und wird vom Brett genommen. Zusammen mit dem auf die selbe Weise
umzingelten Gebiet am Ende des Spieles, bilden die geschlagenen Steine die
Punktzahl der Spieler. Wer mehr Punkte hat gewinnt. Eine Sonderregel ist hier
das Ko. Es ist möglich in eine Endlosschleife zu geraten, indem ein Stein immer
wieder geschlagen und zurückgeschlagen wird. Damit das nicht passiert, darf in
dieser Situation nicht sofort zurückgeschlagen werden, sondern erst einen Zug
später um dem Gegner die Möglichkeit zu geben die Lücke zu schließen. 

Um das Spiel etwas zu vereinfachen und schneller Ergebnisse zu sehen, arbeiten
wir mit $9 {\times} 9$-Spielbrettern. Außerdem gibt es ein Zuglimit für die
Spiele, damit es keine Endlosschleifen gibt, die trotz Ko auftreten können.
Ansonsten haben wir uns aber bemüht die Go Regeln möglichst gut umzusetzen,
damit die Spielstärke auch realistisch beurteilt werden kann.
