\section{Die Idee}
%Was es tun soll, wieso wir das machen, eine kurze Einleitung eben
Neuronale Netze wurden in den letzten Jahren immer wieder für verschiedene
Problemstellungen benutzt. Es wurde festgestellt, dass sie auch komplizierte
Aufgaben lösen können, vorausgesetzt sie werden gut genug trainiert. Go zu
spielen ist ein sehr komplexe Aufgabe, die wir unter anderem genau aus diesem
Grund ausgewählt haben. Wegen der vielen Zugmöglichkeiten und dem mit $19
{\times} 19$ Punkten sehr großen Spielbrett, gibt es für Go noch keine
Computer, die Menschen deutlich übertreffen oder sogar eine perfekte Strategie
spielen können. Dieses Ziel wäre natürlich etwas hoch gegriffen, doch mit
diesem Projekt wollten wir sehen, ob sich neuronale Netze überhaupt gegenseitig
so trainieren können, dass sie bessere Ergebnisse erzielen.
