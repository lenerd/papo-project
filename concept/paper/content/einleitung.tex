\section{Die Idee}
%Was es tun soll, wieso wir das machen, eine kurze Einleitung eben
Neuronale Netze wurden in den letzten Jahren immer wieder für verschiedene
Problemstellungen benutzt. Es wurde festgestellt, dass sie auch komplizierte
Aufgaben lösen können, vorausgesetzt sie werden gut genug trainiert. Go zu
spielen ist ein sehr komplexe Aufgabe, die wir unter anderem genau aus diesem
Grund ausgewählt haben. Wegen der vielen Zugmöglichkeiten und dem mit 19x19
Punkten sehr großen Spielbrett, gibt es für Go noch keine Computer, die
Menschen deutlich übertreffen oder sogar eine perfekte Strategie spielen
können. Dieses Ziel wäre natürlich etwas hoch gegriffen, doch mit diesem
Projekt wollten wir sehen, ob sich neuronale Netze überhaupt gegenseitig so
trainieren können, dass sie bessere Ergebnisse erzielen. Dazu wurden zwei Tools
benutzt, das erste ist dafür zuständig die Netze zu erstellen und mit Hilfe von
Supervised Learning auf die Regeln von Go zu trainieren. Das zweite Tool lässt
die Netze dann gegeneinander antreten und rekombiniert sie mit Hilfe eines
genetischen Algorithmus. Die so entstandenen Netze können gespeichert werden
und auch gegen einen menschlichen Gegner antreten.
