\section{Die Ergebnisse}
%Funktioniert es? Wie ist der Zeitaufwand? Was heißt das für Go mit neuroonalen
%Netzen?hg
Das bisherige Ergebnis war etwas ernüchternd. Die trainierten Netze spielen in
der Regel besser, als untrainierte. Von zehn trainierten Netzen hatte das beste 
eine Gewinnquote von ca. 87\% gegen nicht trainierte Netze, das schlechteste
eine von ca. 45\% und der durchschnitt aller Netze lag bei ca. 67\%.
Doch selbst das beste Netz (egal ob gemessen an der Gewinnquote gegen 
untrainierte oder gleich trainierte) hat keine Chance gegen einen Menschen, 
sofern dieser halbwegs überlegt spielt.

Für die Tragweite von Neuronalen Netzen, im Hinblick auf die Kompetenz Go gut
spielen zu können, hat dies allerdings wenig zu Bedeuten. In diesem Projekt
haben wir lediglich eine von vielen Arten von Neuronalen Netzen benutzt, wobei
nicht klar ist, ob diese Art die best geeignetste ist. Zusätzlich dazu gibt es
in diesem Ansatz viele Parameter, die durch teilweise Abänderung mit Sicherheit 
einen positiven Effekt auf die Performance der Netze hätten. 