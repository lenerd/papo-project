\documentclass[compress]{beamer}

\usetheme{Hamburg}

\usepackage[utf8]{inputenc}
%\usepackage{units}

\title{Projektvorstellung - Kann unser neuronales Netz besser Go spielen als wir?}
\author{Lennart Braun, Armin Schaare, Theresa Eimer}
\institute{Praktikum Parallele Programmierung \\ Arbeitsbereich Wissenschaftliches Rechnen\\Fachbereich Informatik\\Fakultät für Mathematik, Informatik und Naturwissenschaften\\Universität Hamburg}
\date{2015-06-03}

\begin{document}

\begin{frame}
	\titlepage
\end{frame}

\begin{frame}
	\frametitle{Gliederung (Agenda)}

	\tableofcontents
\end{frame}

\section{Go}

\begin{frame}
	\frametitle{Go - Das Spiel}

	\begin{itemize}
		\item Asiatisches Brettspiel
		\item Wird auf Brettern mit 19x19 Knoten gespielt
		\item Ziel: gleichzeitig Gebiet einkreisen und gegnerische Steine schlagen
		\item Spielende: wenn beide Spieler passen
	\end{itemize}

	\begin{figure}
		\begin{center}
			\includegraphics[width=0.75\textwidth]{logo.jpg}
		\end{center}
		\caption{Beispiel eines Bretts}
		\label{fig:logo}
	\end{figure}
\end{frame}

\begin{frame}
	\frametitle{Go - Die Umsetzung}
	
	\begin{itemize}
		\item Gespielt wird auf kleineren Brettern (bis 9x9)
		\item Repräsentation als Datentyp mit dem Brett als Array, einem Array zum Speichern von Gruppen und dem als nächstes ziehenden Spieler
		\item Das Spiel endet, wenn es keine gültigen Züge mehr gibt
		\item Das Spielbrett prüft Züge und verhindert Regelverstöße
		\item Ungültige Spielzüge werden auf den nächstbesten erlaubten Zug gesetzt
	\end{itemize}		
	
\end{frame}

\section{Das Netz}

\begin{frame}[fragile]
	\frametitle{Das Netz}

	\begin{itemize}
		\item Neuronales Netz mit...

		\begin{itemize}
			\item ...n+1 Input-Neuronen für n Knoten auf dem Spielfeld, plus der Differenz schwarzer und weißer Steine
			\item ...beliebig vielen hidden layers
			\item ...2 Output-Neuronen für die x- bzw. y-Koordinate des nächsten Zuges
		\end{itemize}

		\item Ein Neuron gibt sein Signal weiter, wenn das aufsummierte Signal der Neuronen aus der Schicht davor einen Threshhold übersteigen
	\end{itemize}
\end{frame}

\begin{frame}
	\frametitle{Was das Netz kann}
	
	\begin{itemize}
		\item Das Eingangssignal für ein Neuron 	
	\end{itemize}
	
\end{frame}

\section{Der genetische Algorithmus}
\subsection*{}

\begin{frame}
	\frametitle{Zusammenfassung}

	\begin{itemize}
		\item Zusammenfassung 1

		\begin{itemize}
			\item Unterpunkt 1
			\item Unterpunkt 2
		\end{itemize}

		\item Zusammenfassung 2

		\begin{itemize}
			\item Unterpunkt 1
			\item Unterpunkt 2
		\end{itemize}
	\end{itemize}
\end{frame}

\section{Zeitplan}
\subsection*{}

\begin{frame}
	\frametitle{Literatur}

	\nocite*
	\bibliographystyle{alpha}
	\bibliography{literatur}
\end{frame}

\end{document}
