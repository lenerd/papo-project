\pdfminorversion=4
%\documentclass[handout,xcolor=svgnames]{beamer}
\documentclass[presentation,xcolor=svgnames,aspectratio=169]{beamer}

% -*- coding: utf-8 -*-
%UTF-8: äöüß
% !TEX root = ./main.tex
%preface.tex

\usepackage[utf8]{inputenc}
\usepackage[T1]{fontenc}
\usepackage[english,ngerman]{babel}
\usepackage{hyperref}
\usepackage{color}
\usepackage{graphicx}
\usepackage[binary-units]{siunitx}
% \usepackage{rotating}
% \usepackage{xspace} 
% \usepackage{epstopdf}
% \usepackage{pmat}
% \usepackage{tikz}
% \usetikzlibrary{positioning}
% \usetikzlibrary{shapes}

\usepackage{ulem}
\usepackage{cancel}
\usepackage{multicol}

\usepackage{algorithm}
\usepackage{algpseudocode}
\renewcommand{\algorithmicrequire}{\textbf{Input:}}
\renewcommand{\algorithmicensure}{\textbf{Output:}}
\floatname{algorithm}{Algorithmus}
% declaration of the new block
\algblock{ParDo}{EndParDo}
% customising the new block
\algnewcommand\algorithmicparfor{\textbf{for}}
\algnewcommand\algorithmicpardo{\textbf{pardo}}
\algnewcommand\algorithmicendparfor{\textbf{end\ pardo}}
\algrenewtext{ParDo}[1]{\algorithmicparfor\ #1\ \algorithmicpardo}
\algrenewtext{EndParDo}{\algorithmicendparfor}
\algnewcommand{\IIf}[1]{\State\algorithmicif\ #1\ \algorithmicthen}
\algnewcommand{\EndIIf}{\unskip\ \algorithmicend\ \algorithmicif}

\newcommand\pro{\item[\boldmath$+$\unboldmath]}
\newcommand\con{\item[\boldmath$-$\unboldmath]}
\newcommand\unknown{\item[\boldmath$?\,$\unboldmath]}

%%%Einige Einstellungen
\mode<presentation>
{
    \usetheme{CambridgeUS}
    % \usecolortheme[named=DarkGreen]{structure}
    \usecolortheme{seahorse}
    \useinnertheme{default}
    \useoutertheme{infolines}

    % Disable navigation bar
    \setbeamertemplate{navigation symbols}{}

    %\setbeamercovered{transparent}
}


\title{NeuGenGo}
\subtitle{Kann unser neuronales Netz besser Go spielen als wir?}

\author[L. Braun, A. Schaare, T. Eimer]{Lennart Braun, Armin Schaare, Theresa Eimer}

\institute[]
    {Universität Hamburg\\ 
    Fakultät für Mathematik, Informatik und Naturwissenschaften\\
    Fachbereich Informatik, Arbeitsbereich WR\\
    Praktikum Parallele Programmierung SS 15}

\date{9. September 2015}

% PDF Dokumentinformationen: autorenspezifisch
\hypersetup {
    pdfauthor={Lennart Braun, Armin Schaare, Theresa Eimer},
    % pdftitle={\title},
    pdftitle = {\title -- \subtitle},
    pdfsubject = {Universität Hamburg / MIN / FB18 / WR / Praktikum Parallele Programmierung / SS 15},
    pdfkeywords = {parallel programming, neural networks, go},
    plainpages = false, 
    pdfstartpage = {1},
    pdfpagelabels,
    breaklinks = {true},
}

%\usepackage[style=alphabetic, backend=biber]{biblatex}
%\addbibresource{../parallele-algorithmen.bib}

\begin{document}

\maketitle

\input{content/toc.tex}
\section{Problemstellung}

%----------------------------------------------------------------------%SLIDE -
\begin{frame}
    \frametitle{Problemstellung}
    \begin{itemize}
        \item
            Unser Ziel ist es, neuronale Netzwerke zu trainieren, sodass diese
            uns im Go schlagen können.

        \item Zwischenziel / Alternative:
            Können wir neuronale Netze so trainieren, sodass sie besser als
            zufällig erzeugte Netze spielen?
    \end{itemize}

    \hfill \\
    \hfill \\
\end{frame}
%----------------------------------------------------------------------%SLIDE -

%----------------------------------------------------------------------%SLIDE -
\begin{frame}
    \frametitle{Go}
    \begin{columns}
        \column{0.5\textwidth}
        \begin{itemize}
            \item Asiatisches Brettspiel
            \item Wird auf Brettern mit $19 \times 19$ Knoten gespielt.
            \item Ziel: Gebiet einkreisen und gegnerische Steine schlagen
            \item Spielende: wenn beide Spieler passen
        \end{itemize}
        \column{0.5\textwidth}
        \begin{figure}
            \centering
            \includegraphics[scale=0.25]{content/img/go_board}
            \caption{generated with qGo}
        \end{figure}
    \end{columns}
\end{frame}
%----------------------------------------------------------------------%SLIDE -

%----------------------------------------------------------------------%SLIDE -
\begin{frame}
    \frametitle{Neuronale Netzwerke}
    \begin{itemize}
        \item Besteht aus mehreren Schichten (layer)
        \item Layer bestehen aus Neuronen
        \item Neuronen benachbarter layer sind alle durch Kanten untereinander verbunden
        \item Neuronen berechnen ihre Werte durch Aufsummieren aller eingehenden Kantengewichte multipliziert mit den Werten an den ausgehenden Neuronen
        \item Sigmoid funktion wird auf das Ergebnis angewandt, sodass alle Werte zwischen 0 und 1 sind.
    \end{itemize}
    \begin{figure}
        TODO: Graphik (möglichst unter CC / selbst erstellt)
        \caption{Schema eines neuronalen Netzwerks}
    \end{figure}
\end{frame}
%----------------------------------------------------------------------%SLIDE -

%----------------------------------------------------------------------%SLIDE -
\begin{frame}
    \frametitle{Neuronale Netzwerke}
    \begin{figure}
        \centering
        % nnet example
\begin{tikzpicture}
    [
        vertex/.style={circle, scale=1.2, draw,},
        edge/.style={-latex,},
        scale=1.5,
    ]

    \node (I1) at (0,0.5)  [vertex] {};
    \node (I2) at (0,1.5)  [vertex] {};
    \node (IT) at (0,3)    [vertex, scale=0.5] {T};
    \node (B1) at (2,0)    [vertex] {};
    \node (B2) at (2,1)    [vertex] {};
    \node (B3) at (2,2)    [vertex] {};
    \node (BT) at (2,3)    [vertex, scale=0.5] {T};
    \node (C1) at (5,0)    [vertex] {};
    \node (C2) at (5,1)    [vertex] {};
    \node (C3) at (5,2)    [vertex] {};
    \node (CT) at (5,3)    [vertex, scale=0.5] {T};
    \node (O1) at (7,0.5)  [vertex] {};
    \node (O2) at (7,1.5)  [vertex] {};

    \node at (0,4) {Input Layer};
    \node at (3.5,4) {Hidden Layer};
    \node at (7,4) {Output Layer};
    \draw [
        decorate,
        decoration={brace, amplitude=5},
    ] (-0.5,3.5) -- (0.5,3.5);
    \draw [
        decorate,
        decoration={brace, amplitude=5},
    ] (1.5,3.5) -- (5.5,3.5);
    \draw [
        decorate,
        decoration={brace, amplitude=5},
    ] (6.5,3.5) -- (7.5,3.5);

    \draw [edge] (I1) -- (B1);
    \draw [edge] (I1) -- (B2);
    \draw [edge] (I1) -- (B3);
    \draw [edge] (I2) -- (B1);
    \draw [edge] (I2) -- (B2);
    \draw [edge] (I2) -- (B3);
    \draw [edge] (IT) -- (B1);
    \draw [edge] (IT) -- (B2);
    \draw [edge] (IT) -- (B3);

    \draw [
        line width=2,
        line cap=round,
        dash pattern=on 0 off 10,
    ] (3.25,0) -- (3.75,0);
    \draw [
        line width=2,
        line cap=round,
        dash pattern=on 0 off 10,
    ] (3.25,1) -- (3.75,1);
    \draw [
        line width=2,
        line cap=round,
        dash pattern=on 0 off 10,
    ] (3.25,2) -- (3.75,2);
    \draw [
        line width=2,
        line cap=round,
        dash pattern=on 0 off 10,
    ] (3.25,3) -- (3.75,3);

    \draw [dashed] (B1) -- ($ (B1) !.33! (C1) $);
    \draw [edge, dashed] ($ (B1) !.66! (C1) $) -- (C1);
    \draw [dashed] (B1) -- ($ (B1) !.33! (C2) $);
    \draw [edge, dashed] ($ (B1) !.66! (C2) $) -- (C2);
    \draw [dashed] (B1) -- ($ (B1) !.33! (C3) $);
    \draw [edge, dashed] ($ (B1) !.66! (C3) $) -- (C3);

    \draw [dashed] (B2) -- ($ (B2) !.33! (C1) $);
    \draw [edge, dashed] ($ (B2) !.66! (C1) $) -- (C1);
    \draw [dashed] (B2) -- ($ (B2) !.33! (C2) $);
    \draw [edge, dashed] ($ (B2) !.66! (C2) $) -- (C2);
    \draw [dashed] (B2) -- ($ (B2) !.33! (C3) $);
    \draw [edge, dashed] ($ (B2) !.66! (C3) $) -- (C3);

    \draw [dashed] (B3) -- ($ (B3) !.33! (C1) $);
    \draw [edge, dashed] ($ (B3) !.66! (C1) $) -- (C1);
    \draw [dashed] (B3) -- ($ (B3) !.33! (C2) $);
    \draw [edge, dashed] ($ (B3) !.66! (C2) $) -- (C2);
    \draw [dashed] (B3) -- ($ (B3) !.33! (C3) $);
    \draw [edge, dashed] ($ (B3) !.66! (C3) $) -- (C3);

    \draw [dashed] (BT) -- ($ (BT) !.33! (C1) $);
    \draw [edge, dashed] ($ (BT) !.66! (C1) $) -- (C1);
    \draw [dashed] (BT) -- ($ (BT) !.33! (C2) $);
    \draw [edge, dashed] ($ (BT) !.66! (C2) $) -- (C2);
    \draw [dashed] (BT) -- ($ (BT) !.33! (C3) $);
    \draw [edge, dashed] ($ (BT) !.66! (C3) $) -- (C3);

    \draw [edge] (C1) -- (O1);
    \draw [edge] (C1) -- (O2);
    \draw [edge] (C2) -- (O1);
    \draw [edge] (C2) -- (O2);
    \draw [edge] (C3) -- (O1);
    \draw [edge] (C3) -- (O2);
    \draw [edge] (CT) -- (O1);
    \draw [edge] (CT) -- (O2);

\end{tikzpicture}         

        % \caption{Feedforward Netz}
        \label{fig:nnet}
    \end{figure}
\end{frame}
%----------------------------------------------------------------------%SLIDE -

\section{Lösungsansatz}

%----------------------------------------------------------------------%SLIDE -
\begin{frame}
\end{frame}
%----------------------------------------------------------------------%SLIDE -

\section{Parallelisierungsschema}

%----------------------------------------------------------------------%SLIDE -
\begin{frame}
    \frametitle{Parallelisierungsschema}

    Was ist parallelisierbar?
    \begin{itemize}
        \item[--]
            Die Generationen sind inherent sequentiell
        \item[+]
            die einzelnen Spiele sind unabhängig voneinander
            (\textbf{for} $net_a, net_b \in N_i$ \textbf{do})
        \item[+]
            die Ausgabeberechnung in den Neuronalen Netzwerken
            (dreifache Schleife)
    \end{itemize}
\end{frame}
%----------------------------------------------------------------------%SLIDE -

%----------------------------------------------------------------------%SLIDE -
\begin{frame}
    \frametitle{Parallelisierung der Spielphase}
    \framesubtitle{message passing}

    Ziel: $n^2$ Spiele bei $n$ Netzen auf $p$ Prozesse zu verteilen

    notwendige Kommunikation:
    \begin{itemize}
        \item die Netze $n \cdot \SI{1.59}{\kibi\byte}$
        \item Anzahl der Siege $n \cdot \SI{8}{\byte}$ (uint8\_t verwenden?)
    \end{itemize}
    Probleme:
    \begin{itemize}
        \item neue Netzwerke in jeder Runde
        \item unterschiedliche Laufzeit der Spiele \\
            $\Rightarrow$ Wartezeiten bei Ringstruktur
    \end{itemize}
\end{frame}
%----------------------------------------------------------------------%SLIDE -

%----------------------------------------------------------------------%SLIDE -
\begin{frame}
    \frametitle{Parallelisierung der Spielphase}
    \framesubtitle{message passing}

    \begin{columns}
        \column{0.5\textwidth}
        \begin{algorithm}[H]
            \caption{parallele Spielphase}
            \begin{algorithmic}[1]
                \For {Generation $i = 0$ bis $\ldots$}
                    \State \Call{broadcast}{$N_i$, 0}
                    \ParDo {$\forall net_a \in N_i$}
                        \For {$\forall net_b \neq net_a \in N_i$}
                            \State lass $net_a, net_b$ spielen
                            \State zähle die Siege ($wins$)
                        \EndFor
                    \EndParDo
                    \State \Call{reduce}{$wins$}
                    \IIf {rank = 0}
                        generiere $N_{i+1}$
                    \EndIIf
                \EndFor
            \end{algorithmic}
        \end{algorithm}

        \column{0.5\textwidth}
        \begin{itemize}
            \item Jeder Prozess erhält alle Netze.
            \item Gleichmäßige Verteilung der äußeren Schleife (Zeile 3).
        \end{itemize}
    \end{columns}
\end{frame}
%----------------------------------------------------------------------%SLIDE -

%----------------------------------------------------------------------%SLIDE -
\begin{frame}
    \frametitle{Parallelisierung der Spielphase}
    \framesubtitle{Spurdatenanalyse}

    \begin{figure}
        \centering
        TODO: Visualisierung durch Vampir
        \caption{Vampir}
    \end{figure}

\end{frame}
%----------------------------------------------------------------------%SLIDE -

%----------------------------------------------------------------------%SLIDE -
\begin{frame}
    \frametitle{Parallelisierung der neuronalen Netzwerke}
    \framesubtitle{shared memory}

    TODO:
\end{frame}
%----------------------------------------------------------------------%SLIDE -

\section{Die Ergebnisse}
%Funktioniert es? Wie ist der Zeitaufwand? Was heißt das für Go mit neuroonalen
%Netzen?

\section*{Zahlen}

%----------------------------------------------------------------------%SLIDE -
\begin{frame}
    \frametitle{Zahlen}

    TODO: Commits, LoCs, GitHub URI

\end{frame}
%----------------------------------------------------------------------%SLIDE -


\end{document}
