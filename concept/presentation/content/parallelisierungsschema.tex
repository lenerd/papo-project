\section{Parallelisierungsschema}

%----------------------------------------------------------------------%SLIDE -
\begin{frame}
    \frametitle{Parallelisierungsschema}

    Was ist parallelisierbar?
    \begin{itemize}
        \item[--]
            Die Generationen sind inherent sequentiell
        \item[+]
            die einzelnen Spiele sind unabhängig voneinander
            (\textbf{for} $net_a, net_b \in N_i$ \textbf{do})
        \item[+]
            die Ausgabeberechnung in den Neuronalen Netzwerken
            (dreifache Schleife)
    \end{itemize}
\end{frame}
%----------------------------------------------------------------------%SLIDE -

%----------------------------------------------------------------------%SLIDE -
\begin{frame}
    \frametitle{Parallelisierung der Spielphase}
    \framesubtitle{message passing}

    Ziel: $n^2$ Spiele bei $n$ Netzen auf $p$ Prozesse zu verteilen

    notwendige Kommunikation:
    \begin{itemize}
        \item die Netze $n \cdot \SI{1.59}{\kibi\byte}$
        \item Anzahl der Siege $n \cdot \SI{8}{\byte}$ (uint8\_t verwenden?)
    \end{itemize}
    Probleme:
    \begin{itemize}
        \item neue Netzwerke in jeder Runde
        \item unterschiedliche Laufzeit der Spiele \\
            $\Rightarrow$ Wartezeiten bei Ringstruktur
    \end{itemize}
\end{frame}
%----------------------------------------------------------------------%SLIDE -

%----------------------------------------------------------------------%SLIDE -
\begin{frame}
    \frametitle{Parallelisierung der Spielphase}
    \framesubtitle{message passing}

    \begin{columns}
        \column{0.5\textwidth}
        \begin{algorithm}[H]
            \caption{parallele Spielphase}
            \begin{algorithmic}[1]
                \For {Generation $i = 0$ bis $\ldots$}
                    \State \Call{broadcast}{$N_i$, 0}
                    \ParDo {$\forall net_a \in N_i$}
                        \For {$\forall net_b \neq net_a \in N_i$}
                            \State lass $net_a, net_b$ spielen
                            \State zähle die Siege ($wins$)
                        \EndFor
                    \EndParDo
                    \State \Call{reduce}{$wins$}
                    \IIf {rank = 0}
                        generiere $N_{i+1}$
                    \EndIIf
                \EndFor
            \end{algorithmic}
        \end{algorithm}

        \column{0.5\textwidth}
        \begin{itemize}
            \item Jeder Prozess erhält alle Netze.
            \item Gleichmäßige Verteilung der äußeren Schleife (Zeile 3).
        \end{itemize}
    \end{columns}
\end{frame}
%----------------------------------------------------------------------%SLIDE -

%----------------------------------------------------------------------%SLIDE -
\begin{frame}
    \frametitle{Parallelisierung der Spielphase}
    \framesubtitle{Spurdatenanalyse}

    \begin{figure}
        \centering
        TODO: Visualisierung durch Vampir
        \caption{Vampir}
    \end{figure}

\end{frame}
%----------------------------------------------------------------------%SLIDE -

%----------------------------------------------------------------------%SLIDE -
\begin{frame}
    \frametitle{Parallelisierung der neuronalen Netzwerke}
    \framesubtitle{shared memory}

    TODO:
\end{frame}
%----------------------------------------------------------------------%SLIDE -
