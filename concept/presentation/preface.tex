% -*- coding: utf-8 -*-
%UTF-8: äöüß
% !TEX root = ./main.tex
%preface.tex

\usepackage[utf8]{inputenc}
\usepackage[T1]{fontenc}
\usepackage[english,ngerman]{babel}
\usepackage{hyperref}
\usepackage{color}
\usepackage{graphicx}
\usepackage[binary-units]{siunitx}
% \usepackage{rotating}
% \usepackage{xspace} 
% \usepackage{epstopdf}
% \usepackage{pmat}
\usepackage{tikz}
\usetikzlibrary{decorations.pathreplacing}
\usetikzlibrary{calc}
% \usetikzlibrary{positioning}
% \usetikzlibrary{shapes}

\usepackage{ulem}
\usepackage{cancel}
\usepackage{multicol}

\usepackage{algorithm}
\usepackage{algpseudocode}
\renewcommand{\algorithmicrequire}{\textbf{Input:}}
\renewcommand{\algorithmicensure}{\textbf{Output:}}
\floatname{algorithm}{Algorithmus}
% declaration of the new block
\algblock{ParDo}{EndParDo}
% customising the new block
\algnewcommand\algorithmicparfor{\textbf{for}}
\algnewcommand\algorithmicpardo{\textbf{pardo}}
\algnewcommand\algorithmicendparfor{\textbf{end\ pardo}}
\algrenewtext{ParDo}[1]{\algorithmicparfor\ #1\ \algorithmicpardo}
\algrenewtext{EndParDo}{\algorithmicendparfor}
\algnewcommand{\IIf}[1]{\State\algorithmicif\ #1\ \algorithmicthen}
\algnewcommand{\EndIIf}{\unskip\ \algorithmicend\ \algorithmicif}

\newcommand\pro{\item[\boldmath$+$\unboldmath]}
\newcommand\con{\item[\boldmath$-$\unboldmath]}
\newcommand\unknown{\item[\boldmath$?\,$\unboldmath]}

%%%Einige Einstellungen
\mode<presentation>
{
    \usetheme{CambridgeUS}
    % \usecolortheme[named=DarkGreen]{structure}
    \usecolortheme{seahorse}
    \useinnertheme{default}
    \useoutertheme{infolines}

    % Disable navigation bar
    \setbeamertemplate{navigation symbols}{}

    %\setbeamercovered{transparent}
}
